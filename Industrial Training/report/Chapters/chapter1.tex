\chapter{Introduction}

You'll probably get ten different replies if you ask ten different mobile developers how they design their apps for Android or iOS devices. Because different languages are used for different operating systems, development teams for each operating system, such as Android, iOS, Windows, Mac, Linux, and web-apps, are required. This has a significant impact on the project's construction costs. As with every project, multiple meetings, development phases, testing phases, designing phases, and different codebases are required, and these softwares are difficult to manage because each change in the software must be reflected in each and every code base individually by the various teams.

To avoid this reason there was a need for cross platform development from the old-fashioned porting of apps. For solving this problem, Google developed the flutter framework to make cross platform development easier, with the view of - ’One codebase, multiple platforms!’. 


\section{History}

The most popular version of Flutter was codenamed "Sky" and operated on the Android operating system. It was announced during the 2015 Dart engineer summit with the stated goal of being able to dependably deliver at 120 frames per second. During the September 2018 Google Developer Days in Shanghai, Google announced Flutter Release Preview 2, the final major release before Flutter 1.0. Flutter 1.0, the main "stable" edition of the Framework, was released on December 4th of that year at the Flutter Live event. At the Flutter Interactive event on December 11, 2019, Flutter 1.12 was released.

The Dart programming advancement unit (SDK) in version 2.8 and Flutter in version 1.17.0 were released on May 6, 2020, with support for the Metal API, which improved execution on iOS devices (by about half), new Material gadgets, and new organisation following.

Google released Flutter 2 on March 3rd, 2021, during an online Flutter Engage event. This major release added official support for online apps, including a new Canvas Kit renderer and web explicit gadgets, as well as early-access desktop application support for Windows, macOS, and Linux, and improved Add-to-App APIs. The Flutter group supplied directions to relieve these progressions as well. This delivery incorporated sound invalid wellness, which created several breaking changes and troubles with multiple exterior bundles, but the Flutter group gave directions to alleviate these progressions as well.

Google released the Dart SDK in version 2.14 and Flutter adaption 2.5 on September 8, 2021. The update included improvements to Android Full-Screen mode and Material You, Google's most recent Material Design version. Dart received two new upgrades, with the most recent build conditions being normalised and set as the defaults. Dart for Apple Silicon is currently stable.
