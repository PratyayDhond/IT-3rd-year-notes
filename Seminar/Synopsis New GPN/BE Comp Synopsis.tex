\documentclass[11pt,a4paper,oneside,openright]{report}
%\documentclass[11pt,a4paper,openright]{report}
\usepackage{anysize}
\usepackage{vpcoestyle}   %% Thesis Title Page
\usepackage{latexsym}
%\usepackage{pictex}
\usepackage{fancyvrb}
\usepackage{fancybox}
\usepackage{fancyhdr}
\usepackage{color}
\usepackage{setspace}
\usepackage{amssymb}
\usepackage{multicol}
\usepackage{graphicx}
\usepackage{subfigure}
\usepackage{rawfonts}
\usepackage{setspace}
%\usepackage{newcent} %the options are:  palatino, times, newcent, bookman

\marginsize{1.5in}{1in}{0.75in}{0.75in}

\usepackage[pdftex,
        colorlinks=true,
        urlcolor=black,       % \href{...}{...} external (URL)
        filecolor=black,     % \href{...} local file
        linkcolor=black,       % \ref{...} and \pageref{...}
	citecolor=black,
        pdftitle={Seminar Synopsis},
        pdfauthor={Pratyay Dhond},
        pdfsubject={Seminar},
        pdfkeywords={Flutter},
%        pagebackref,
%        pdfpagemode=true,
        bookmarksopen=true]{hyperref}
\pdfcompresslevel=9


\thisfancypage{%
  \setlength{\fboxsep}{14pt}\shadowbox}{}

\title{Native Application Development using flutter}
\author{Mast. Pratyay Dhond}
\date{OCT 26, 2021}

\begin{document}
\def\what{SYNOPSIS}
\def\title{Native Application Development using flutter}

\def\degree{{\bf \large Diploma in Engineering\\(Information Technology)}}
\def\first{\large  Mast. Pratyay Prasad Dhond }
\def\one{\Large 1907011}

\def\guide{\Large Dr. A R Mahajan}
\def\coguide{Prof.XYZ }
\pagenumbering{roman}

%\ifpdf
\pdfbookmark[1]{Title}{title} %%This is pdf specefic
%\fi

%--------------Title Page-------------
\vpcoetitlepage
\cleardoublepage
\thispagestyle{empty}
\thisfancypage{%
  \setlength{\fboxsep}{14pt}\shadowbox}{}
\begin{center} \label{chap:approval}
{\bf \huge {Approval Sheet}} \\

\vspace{0.8in}
\end{center}
\renewcommand{\baselinestretch}{1.5}
\hspace{0.2in} \large {The synopsis entitled {\bf Native Application Development using Flutter} submitted by \par{\hspace{0.4in}{\bf Mast. Pratyay Prasad Dhond}. {\bf Roll No:1907011}\par{\hspace{0.4in}\\ is approved for the partial fulfilment of the requirement for the award of {\bf Diploma} in {\bf Information Technology}}\\

\vspace{1.5in}
\begin{multicols}{2}
\begin {center}
 {\bf Dr. A R Mahajan}\\ Guide\\%\\Department of Computer Engineering\\.\\
 {\bf Prof.L.D.Vilhekar}\\Project Coordinator\\%\\Department of Computer Engineering\\.\\
\end {center}
\end{multicols}
\vskip 0.6in
\begin{center}
\begin{multicols}{2}

.\\{\bf Seal/Stamp}\\
{\bf \large Dr. A.R.Mahajan}\\{\bf HOD}\\
\end{multicols}
\end{center}


\newpage

%

%%---------------- A B S T R A C T -----------------
\onehalfspacing
\cleardoublepage
\vspace*{2in}
\section*{\begin{center} {\Huge{Abstract}}\end{center}}
%\thispagestyle{empty}
\addcontentsline{toc}{chapter}{\quad\,\,{Abstract}}

 \hspace{0.2in}Flutter is a cross-platform User Interface development framework that is used to develop cross platform applications using a single code base. Flutter can be used to build beautiful, natively compiled application for various platforms such as Android, iOS, Windows, Mac, Linux, Web, etc. using a single codebase. 

 \hspace{0.2in}The key features of the Flutter Framework are Fast Development using the Hot Reload, Expressive and Flexible User Interface(UI) and the Native Performance.

\hspace{0.2in}Fast Development : Flutter's hot reload helps you quickly and easily experiment, build UIs, add features, and fix bugs faster. Experience sub-second reload times without losing state on emulators, simulators, and hardware.

\hspace{0.2in} Expressive, beautiful UIs: Flutter comes with easy to build, highly customizable UI widgets which can be used for easy development. Flutter also supports custom widget building using other widgets.

\hspace{0.2in} Native Performance : Flutter’s widgets incorporate all critical platform differences such as scrolling, navigation, icons and fonts to provide full native performance on both iOS and Android. 

%-----------------Table of Contents-------------------
%\renewcommand{\tableofcontents}{Table of Contents}
\newpage
%\ifpdf
\pdfbookmark[1]{Table of Contents}{contents} %%This is pdf specefic
%\fi
\tableofcontents
%\thispagestyle{empty}
\addtocontents{toc}{\protect\thispagestyle{empty}}
\newpage



%----------------------- Chapter 1 ---------------------
\chapter{Introduction}
\label{chap:intro}
\pagenumbering{arabic}

\section{Introduction}
\hspace{0.2in} If you ask a ten different mobile developers how they develop their mobile applications for Android or iOS devices, you'll probably get 10 different answers. This use of different languages for different Operating systems creates the need of employment of development teams for each operating system like Android, iOS, Windows, Mac, Linux, as well as web-apps,etc. This severely affects the cost of the project to build. 
				
\hspace{0.2in} To avoid this reason there was a need for cross platform development from the old-fashioned porting of apps. For solving this problem, Google developed the flutter framework to make cross platform development easier, with the view of -'One codebase, multiple platforms!'.
			

\begin{itemize}

\item {\bf History of Flutter : }
\end{itemize}
\hspace{0.2in} Flutter began it's life in the year 2017 under the name 'sky' at the dart developer summit.At first, it ran only on google's own own Android operating system, but before long was ported to Apple's ios. 

\hspace{0.2in} Various preveiw versions of Flutter were released subsequent to its initial announcement, culminating in the December  4th,2018, release of Flutter 1.0, the first "Stable" release. One of the main goals of Flutter as stated was, 'being able to render app UIs at consistent 120fps no matter what."

\hspace{0.2in} Fltter iffers twi sets if wudgets : Material design widgets and Cupertino design widgets.

\section{ACM Keywords}
 \begin{enumerate}
 \item UI : User Interface
 \end{enumerate}

%\cleardoublepage

\chapter{Problem Statement}\label{chap:project outline}

\section{Problem Statement}
 Native Application Development using Flutter.
\section{Solving Approach}
The website is to be developed using ASP.Net as the Front End and SQL Server as the Back End.
 C Sharp will be used for coding and ADO.Net providing the classes for database connectivity.

\section{Outcomes}
\hspace{0.2in} These are the different tags which will be included in the Seminar :
\begin{enumerate}
\item Home :
\begin{itemize}
\item Flutter
\item Dart
\item Native App
\item OS
\item Web
\item Cross Platform
\item Contents
\end{itemize}

\item History :  \\
This will contain the details of Flutter such as its history and the timeline of flutter.

\item Importance :\\
\hspace{0.2in}It contains the reasons why flutter is important.

\item Photo Gallery :\\
\hspace{0.2in}It contain various photos such as the logos of flutter as well as dart, and other images and their descriptions, etc.


\item News :\\
\hspace{0.2in}It contains the information about the latest updates, news and features added in dart as well as flutter.

\item Overview:\\
\hspace{0.2in} This part gives a short overview of the whole presentation.

\item My opinion :\\
\hspace{0.2in} This part contains my personal opinion about the future of Dart and Flutter.

\end{enumerate}





\chapter{References}
\begin{enumerate}

\item https://flutter.dev/\\
{\em "This is the official website of Flutter."}

\item https://pub.dev/\\
{\em "This is a website where developers can access various open source dependencies and implement them in their project."}


\item Practical Flutter - Frank Zammetti\\
{\em "This is a great book for anyone who wants to learn to code using flutter framework and build cross platform apps."}

\end{enumerate}










\end{document}
 