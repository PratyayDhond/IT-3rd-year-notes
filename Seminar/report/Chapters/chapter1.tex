\chapter{Introduction}

If you ask ten different mobile developers how they develop their mobile applications for Android or iOS devices, you’ll probably get 10 different answers. This use of different languages for different Operating systems creates the need of employment of development teams for each operating system like Android, iOS, Windows, Mac, Linux, as well as web-apps,etc. This severely affects the cost of the project to build. As with every project to be built there is a need for different meetings, development phases, testing phases, designing phases, different codebases and ultimately these softwares are hard to manage as a change in the software will need the change to be reflected in each and every code base individually by the different teams.

To avoid this reason there was a need for cross platform development from
the old-fashioned porting of apps. For solving this problem, Google developed the
flutter framework to make cross platform development easier, with the view of
-’One codebase, multiple platforms!’. 



\section{History}

The principal adaptation of Flutter was known by the codename "Sky" and ran on the Android operating framework. It was revealed at the 2015 Dart engineer summit with the expressed expectation of having the option to deliver reliably at 120 frames per second. During the feature of Google Developer Days in Shanghai in September 2018, Google declared Flutter Release Preview 2, which is the last large delivery before Flutter 1.0. On December fourth of that year, Flutter 1.0 was delivered at the Flutter Live occasion, meaning the principal "stable" variant of the Framework. On December 11, 2019, Flutter 1.12 was delivered at the Flutter Interactive event.

On May 6, 2020, the Dart programming advancement unit (SDK) in variant 2.8 and the Flutter in form 1.17.0 were delivered, where backing was added to the Metal API, further developing execution on iOS gadgets (around half), new Material gadgets, and new organization following.

On March 3, 2021, Google delivered Flutter 2 during an internet based Flutter Engage occasion. This significant update brought official help for online applications with new Canvas Kit renderer and web explicit gadgets, early-access desktop application support for Windows, macOS, and Linux and further developed Add-to-App APIs. This delivery included sound invalid wellbeing, which caused many breaking changes and issues with numerous outer bundles, yet the Flutter group included directions to alleviate these progressions also.

On September eighth, 2021, the Dart SDK in variant 2.14 and Flutter adaptation 2.5 were delivered by Google. The update carried upgrades to the Android Full-Screen mode and the most recent variant of Google's Material Design called Material You. Dart got two new updates, the most up to date build up conditions have been normalized and preset as the default conditions too Dart for Apple Silicon is presently steady.
