\chapter{Comparision to other development platforms}




\section{Apple and Android Development}
Native applications offer the least trouble in adopting new features. These Native Applications tend to have user
experiences more in tune with the given platform since the applications are built specifically for the platforms, using the controls from
the platform vendors themselves (Apple or Google) and often they follow design guidelines set out by
these vendors.\\
One big advantage of the native applications is that they can adopt brand new technologies, make changes, and access the APIs for the platform directly with the help of lots of documentation available for the Native application Development framework. Apple
and Google create features available in beta immediately if desired, without having to wait for any third-party
integration. The main disadvantage to building native applications is the lack of code reuse, as the code is specific to the platform, that is it cannot be used across
platforms, which can make development expensive if targeting iOS and Android.

\section{React Native}
React Native allows the software developers to build native applications using JavaScript. The actual controls the
application uses are native platform controls, so the end user gets the feel of a native app. For apps
that require customization beyond what React Native’s abstraction provides, native development could
still be needed.\\ In cases where the amount of customization required is substantial, the benefit of
working within React Native’s abstraction layer lessens to the point where in some cases developing
the app natively would be more beneficial